\begin{figure}
\centering
\begin{tikzpicture}[
plain/.style={
  draw=none,
  fill=none,
  },
net/.style={
  matrix of nodes,
  nodes={
    draw,
    circle,
    inner sep=5pt
    },
  nodes in empty cells,
  column sep=-0.5cm,
  row sep=-5pt
  },
>=latex
]
\matrix[net] (mat)
{
|[plain]| \parbox{1.3cm}{\centering \hfill} & |[plain]| \parbox{1.3cm}{Encode} & |[plain]| \parbox{1.3cm}{\centering \hfill}  & |[plain]| \parbox{1.3cm}{\raggedleft Decode} & |[plain]| \parbox{1.3cm}{\centering \hfill} \\
& & |[plain]| & & &|[plain]|\\
|[plain]| & |[plain]|& & |[plain]|\\
& &|[plain]| & & &|[plain]|\\
|[plain]| & |[plain]| \\
&& & & &|[plain]|\\
|[plain]| &|[plain]|& |[plain]| \\
& & |[plain]| & & &|[plain]|\\
|[plain]|&|[plain]| & & |[plain]|\\
& & |[plain]| & & &|[plain]|\\
};
\foreach \ai [count=\mi ]in {2,4,...,11}
  \draw[<-] (mat-\ai-1) -- node[above] {$x_\mi$} +(-2cm,0);

% Middle lines  
\foreach \ai in {2,4,...,10}
{\foreach \aii in {2,4,8,10}
  \draw[->] (mat-\ai-1) -- (mat-\aii-2);
}

\foreach \ai in {2,4,...,10}
{\foreach \aii in {3,6,9}
  \draw[->] (mat-\ai-2) -- (mat-\aii-3);
}
% last three lines
\foreach \ai in {2,4,6,8,10}
{
\foreach \aii in {3,6,9}
  \draw[->] (mat-\aii-3) -- (mat-\ai-4);
}

\foreach \ai in {2,4,...,10}
{\foreach \aii in {2,4,8,10}
  \draw[->] (mat-\ai-4) -- (mat-\aii-5);
}


\foreach \ai [count=\mi ]in {2,4,...,11}
  \draw[->] (mat-\ai-5) -- node[above] {$y_\mi$} +(2cm,0);
\end{tikzpicture}
\caption{
A simple autoencoder with a two layer encoder/decoder pair. This autoencoder
would code a five dimensional space into a three dimensional space. Each neuron represents a weighted sum passed through an activation function.
}
\label{autoenc}
\end{figure}

An autoencoder is a particular type of neural network. A neural network is a computational graph where nodes can be
defined in the form: 
\begin{equation}
\mathbf{y} = f\left(W\mathbf{x} + \mathbf{b}\right)
\label{dense}
\end{equation}
Where $\mathbf{x}$ is an input vector, $W$ and $b$ are parameters that define a
linear relationship, and $f(\cdot)$ is an activation function. The activation
function is generally a non-linear function so each node in the graph can
perform a non-linear mapping from $\mathbf{x}$ to $\mathbf{y}$.

Nodes can be connected successively to one another in a feed-forward fashion to
create increasingly complex representations of the input data. Nodes in these
networks are generally referred to as layers, and can be represented
diagrammatically as in Figure \ref{autoenc}. 

Figure \ref{autoenc} in particular shows an autoencoder. Unlike an ordinary
neural network an autoencoder contains a bottleneck in the middle. It is this
bottleneck that creates the rich dimensionality reduction power of an
autoencoder. This bottleneck essentially splits the network into two stages:
the encoder stage and the decoder stage. We interpret the autoencoder as
having two stages because we optimize the learned parameters of the
autoencoder to minimize the reconstruction error at the output of the decoder.
By following this objective and applying other constraints, such as a sparsity constraints, rich coding schemes can be achieved at the bottleneck.
